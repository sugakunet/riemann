\chapter{de Branges Theory}
\label{chap:debranges}

\begin{quote}
\textit{``The theory of Hilbert spaces of entire functions provides a natural framework for the spectral interpretation of the Riemann Hypothesis, though fundamental gaps remain in establishing the required positivity conditions.''}
\end{quote}

\section{Introduction}

The theory of Hilbert spaces of entire functions, developed by Louis de Branges in the 1960s, represents one of the most sophisticated attempts to prove the Riemann Hypothesis through operator-theoretic methods. This approach seeks to realize the zeros of the zeta function as eigenvalues of a self-adjoint operator, thereby reducing the Riemann Hypothesis to a spectral positivity condition.

While the de Branges approach has not succeeded in proving RH, it has created powerful mathematical tools with applications far beyond number theory, including the solution of the Hamburger moment problem and advances in inverse spectral theory. This chapter presents both the remarkable achievements of the theory and the significant obstacles that have prevented its application to RH.

\section{Hilbert Spaces of Entire Functions}
\label{sec:hilbert-spaces-entire}

\subsection{Core Definition and Axioms}

\begin{definition}[de Branges Space]
A \textbf{de Branges space} $\mathcal{H}(E)$ is a Hilbert space of entire functions satisfying three fundamental axioms:
\end{definition}

\begin{axiom}[Zero Removal Axiom (H1)]
If $F(z) \in \mathcal{H}(E)$ has a nonreal zero $w$, then 
$$\frac{F(z)(z-\bar{w})}{z-w} \in \mathcal{H}(E)$$
with the same norm as $F$.
\end{axiom}

\begin{axiom}[Point Evaluation Axiom (H2)]
For every nonreal $w \in \mathbb{C}$, the evaluation functional $F \mapsto F(w)$ is continuous on $\mathcal{H}(E)$.
\end{axiom}

\begin{axiom}[Conjugation Axiom (H3)]
If $F(z) \in \mathcal{H}(E)$, then $F^*(z) = \overline{F(\bar{z})} \in \mathcal{H}(E)$ with $\|F^*\| = \|F\|$.
\end{axiom}

These axioms encode essential properties that connect function theory with operator theory. The zero removal axiom (H1) allows systematic study of zero distributions, while axioms (H2) and (H3) ensure compatibility with complex analysis and provide the necessary structure for spectral interpretations.

\subsection{Structure Functions}

\begin{definition}[Structure Function]
A de Branges space $\mathcal{H}(E)$ is generated by a \textbf{structure function} $E(z) = A(z) - iB(z)$ where:
\begin{enumerate}
\item $A(z)$ and $B(z)$ are real entire functions, real on $\mathbb{R}$
\item $|E(\bar{z})| < |E(z)|$ for $\text{Im}(z) > 0$
\item $E(z)$ has no real zeros
\end{enumerate}
\end{definition}

The structure function $E(z)$ completely determines the space $\mathcal{H}(E)$ and its norm structure. The condition $|E(\bar{z})| < |E(z)|$ in the upper half-plane is crucial for ensuring that the space has the correct analytic properties.

\begin{definition}[Norm in de Branges Space]
The norm in $\mathcal{H}(E)$ is defined by:
$$\|F\|^2 = \int_{-\infty}^{\infty} \left|\frac{F(t)}{E(t)}\right|^2 dt$$
for functions $F \in \mathcal{H}(E)$ such that $F/E \in L^2(\mathbb{R})$.
\end{definition}

\subsection{Growth Estimates}

Functions in de Branges spaces satisfy fundamental growth constraints that control their behavior in the complex plane.

\begin{theorem}[Fundamental Inequality]
\label{thm:fundamental-inequality}
For $F \in \mathcal{H}(E)$ and $z \in \mathbb{C}$ with $\text{Im}(z) \neq 0$:
$$|F(z)|^2 \leq \|F\|^2 \cdot \frac{|E(z)|^2 - |E(\bar{z})|^2}{4\pi |\text{Im}(z)|}$$
\end{theorem}

This inequality provides essential control over the growth of functions in the space and is fundamental to many applications of the theory.

\begin{example}[Paley-Wiener Space]
The classical Paley-Wiener space $PW_\pi$ of entire functions of exponential type at most $\pi$ that are square-integrable on the real line is a de Branges space with structure function $E(z) = e^{i\pi z}$.
\end{example}

\section{Reproducing Kernel Structure}
\label{sec:reproducing-kernel}

\subsection{The Reproducing Kernel Formula}

Every de Branges space possesses a reproducing kernel that encodes its geometric structure.

\begin{theorem}[Reproducing Kernel]
\label{thm:reproducing-kernel}
The reproducing kernel for $\mathcal{H}(E)$ is given by:
$$K(w,z) = \frac{B(z)A(\bar{w}) - A(z)B(\bar{w})}{\pi(z-\bar{w})}$$
where $E(z) = A(z) - iB(z)$.
\end{theorem}

\begin{proof}
The kernel satisfies the defining properties:
\begin{enumerate}
\item $K(w,\cdot) \in \mathcal{H}(E)$ for all $w \in \mathbb{C}$
\item $F(w) = \langle F, K(w,\cdot) \rangle$ for all $F \in \mathcal{H}(E)$
\item $\|K(w,\cdot)\|^2 = K(w,w)$
\end{enumerate}
The verification follows from the axioms (H1)-(H3) and the definition of the norm.
\end{proof}

\subsection{Properties and Applications}

The reproducing kernel provides a powerful tool for studying the geometry of de Branges spaces:

\begin{proposition}[Kernel Properties]
\begin{enumerate}
\item $K(w,z) = \overline{K(z,w)}$ (Hermitian symmetry)
\item $K(w,w) > 0$ for all $w \in \mathbb{C}$ (positive definiteness)
\item The kernel determines the space uniquely
\end{enumerate}
\end{proposition}

The reproducing property allows explicit computation of point evaluations and provides a concrete realization of the abstract Hilbert space structure.

\section{Ordering and Inclusion Theory}
\label{sec:ordering-inclusion}

\subsection{The Chain Theorem}

One of the most remarkable features of de Branges theory is the existence of a natural ordering on spaces.

\begin{theorem}[Chain Theorem]
\label{thm:chain-theorem}
Let $\mathcal{H}(E_1)$ and $\mathcal{H}(E_2)$ be de Branges spaces, both contained isometrically in a third de Branges space $\mathcal{H}(E_3)$. Then either:
\begin{enumerate}
\item $\mathcal{H}(E_1) \subseteq \mathcal{H}(E_2)$, or
\item $\mathcal{H}(E_2) \subseteq \mathcal{H}(E_1)$
\end{enumerate}
\end{theorem}

This creates a \textbf{total ordering} on de Branges spaces contained in a given space, which is fundamental to the classification theory.

\subsection{Characterization Theorem}

The following result shows that the axioms (H1)-(H3) completely characterize de Branges spaces:

\begin{theorem}[Characterization of de Branges Spaces]
\label{thm:characterization}
Every Hilbert space of entire functions satisfying axioms (H1), (H2), and (H3) is isometrically equal to some $\mathcal{H}(E)$ for an appropriate structure function $E$.
\end{theorem}

This theorem establishes that the abstract axioms have a concrete realization in terms of structure functions, providing both existence and uniqueness for the theory.

\subsection{Isometric Embeddings}

The inclusion relationships between de Branges spaces can be characterized explicitly:

\begin{proposition}[Inclusion Criterion]
$\mathcal{H}(E_1) \subseteq \mathcal{H}(E_2)$ isometrically if and only if there exists an entire function $\alpha(z)$ such that:
$$E_1(z) = \alpha(z)E_2(z)$$
and $|\alpha(z)| \leq 1$ on $\mathbb{R}$.
\end{proposition}

\section{Connection to Krein Theory}
\label{sec:krein-connection}

\subsection{Entire Operators and n-Entire Operators}

The de Branges theory is intimately connected to M.G. Krein's theory of entire operators.

\begin{definition}[n-Entire Operator]
A symmetric operator $A$ with deficiency indices $(1,1)$ is called \textbf{n-entire} if its resolvent $(A-z)^{-1}$ has specific growth properties characterized by the parameter $n$.
\begin{enumerate}
\item $n = 0$: Krein's original entire operators
\item $n = -\infty$: Jacobi (tridiagonal) operators  
\item $n \in \mathbb{Z}$: Intermediate classes
\end{enumerate}
\end{definition}

\subsection{Functional Models}

The connection between the theories is established through functional models:

\begin{theorem}[Functional Model Connection]
\label{thm:functional-model}
Every symmetric operator with deficiency indices $(1,1)$ generates a de Branges space $\mathcal{H}(E)$ in which:
\begin{enumerate}
\item The operator acts as multiplication by $z$
\item Self-adjoint extensions correspond to boundary conditions
\item The spectrum is encoded in the structure function $E$
\end{enumerate}
\end{theorem}

\subsection{Classification Hierarchy}

The relationship between different classes of operators forms a hierarchy:
$$\cdots \subset E_{-1}(\mathcal{H}) \subset E_0(\mathcal{H}) \subset E_1(\mathcal{H}) \subset \cdots \subset S(\mathcal{H})$$
where $E_n(\mathcal{H})$ denotes the class of $n$-entire operators and $S(\mathcal{H})$ represents all symmetric operators with deficiency indices $(1,1)$.

\begin{definition}[Multiplication Operator]
In a de Branges space $\mathcal{H}(E)$, the operator $M_z$ of multiplication by $z$ is defined by:
$$(M_z F)(w) = w F(w)$$
for $F \in \mathcal{H}(E)$ and appropriate domain restrictions.
\end{definition}

\begin{proposition}[Properties of $M_z$]
The multiplication operator $M_z$ in $\mathcal{H}(E)$:
\begin{enumerate}
\item Is symmetric with deficiency indices $(1,1)$
\item Has self-adjoint extensions parametrized by $[0,\pi)$  
\item Each extension corresponds to a boundary condition at infinity
\end{enumerate}
\end{proposition}

\section{Application to the Riemann Hypothesis}
\label{sec:rh-application}

\subsection{de Branges' Strategy}

The application of de Branges theory to the Riemann Hypothesis follows a systematic approach:

\begin{strategy}[de Branges' Approach to RH]
\begin{enumerate}
\item Associate to each Dirichlet L-function $L(s,\chi)$ a de Branges space $\mathcal{H}(E_\chi)$
\item Establish that certain positivity conditions hold in these spaces
\item Prove that these conditions imply all zeros lie on the critical line Re$(s) = 1/2$
\end{enumerate}
\end{strategy}

\subsection{Key Construction for the Zeta Function}

For the Riemann zeta function, the construction involves finding a structure function $E(z)$ such that:
$$\xi(1/2 + iz) = c \cdot \frac{E(z)}{E(-z)}$$
where $\xi(s) = \pi^{-s/2}\Gamma(s/2)\zeta(s)$ is the completed zeta function and $c$ is a suitable constant.

\begin{remark}[Physical Interpretation]
This construction attempts to realize the zeta zeros as eigenvalues of a self-adjoint operator, making the Riemann Hypothesis equivalent to the statement that all eigenvalues are real.
\end{remark}

\subsection{Positivity Conditions}

The heart of de Branges' approach lies in establishing positivity conditions:

\begin{conjecture}[de Branges Positivity]
For appropriate test functions $\varphi$ in the constructed de Branges spaces, the sum
$$\sum_{\rho} \varphi(\rho) \geq 0$$
where the sum is over all non-trivial zeros $\rho$ of $\zeta(s)$.
\end{conjecture}

\begin{theorem}[RH Equivalence]
If the de Branges positivity conditions hold, then the Riemann Hypothesis is true.
\end{theorem}

\subsection{Spectral Interpretation}

The spectral interpretation makes RH a statement about the reality of eigenvalues:

\begin{itemize}
\item \textbf{Zeros} correspond to eigenvalues of a self-adjoint operator
\item \textbf{Critical line} Re$(s) = 1/2$ corresponds to real spectrum
\item \textbf{RH} becomes: ``all eigenvalues are real''
\end{itemize}

This reframes a complex analytic problem as a question in spectral theory, potentially opening new avenues for attack.

\section{The Conrey-Li Gap and Technical Challenges}
\label{sec:conrey-li-gap}

\subsection{The Gap Identified by Conrey-Li (2000)}

Despite the elegance of de Branges' approach, a significant gap was identified by Brian Conrey and Xian-Jin Li in 2000.

\begin{theorem}[Conrey-Li Gap]
\label{thm:conrey-li-gap}
The positivity conditions required in de Branges' approach to the Riemann Hypothesis have been proven \textbf{not to hold} in the constructed spaces.
\end{theorem}

This represents a fundamental obstacle to completing the de Branges program for proving RH.

\subsection{Why Positivity Conditions Fail}

The failure of positivity conditions stems from several technical issues:

\begin{problem}[Non-constructive Elements]
The structure functions $E_\chi(z)$ required for the construction are defined through existence theorems rather than explicit formulas, making verification of their properties extremely difficult.
\end{problem}

\begin{problem}[Convergence Issues]
The limiting procedures used to construct the relevant de Branges spaces involve subtle convergence questions that have not been rigorously justified.
\end{problem}

\begin{problem}[Boundary Conditions]
The self-adjoint extensions of the multiplication operator may not be well-defined due to boundary behavior at infinity.
\end{problem}

\subsection{Computational Obstacles}

The de Branges approach faces significant computational challenges:

\begin{itemize}
\item \textbf{Explicit computation}: The relevant de Branges spaces are difficult to work with computationally
\item \textbf{Numerical verification}: Checking positivity conditions numerically is extremely challenging  
\item \textbf{Analytic continuation}: The connection to L-functions involves complex analytic continuation procedures
\end{itemize}

\subsection{Current Status and Assessment}

\begin{assessment}[Current State of de Branges Approach]
As of 2024, the de Branges approach to RH faces the following situation:
\begin{enumerate}
\item The \textbf{theoretical framework} remains mathematically sound and profound
\item The \textbf{main gap} identified by Conrey-Li has not been closed
\item \textbf{No clear path} exists for overcoming the technical obstacles
\item The approach may require \textbf{fundamentally new insights} to proceed
\end{enumerate}
\end{assessment}

\section{Strengths and Limitations}
\label{sec:strengths-limitations}

\subsection{Strengths of the de Branges Approach}

Despite the obstacles to proving RH, the de Branges theory has remarkable strengths:

\begin{strength}[Deep Theoretical Framework]
The theory provides a unified view connecting:
\begin{itemize}
\item Function theory and operator theory
\item Spectral analysis and complex analysis  
\item Classical analysis and modern functional analysis
\end{itemize}
\end{strength}

\begin{strength}[Successful Applications]
The theory has achieved complete success in other areas:
\begin{itemize}
\item Complete solution of the Hamburger moment problem
\item Classification of quantum mechanical operators
\item Advances in inverse spectral problems
\item Solution of various interpolation problems
\end{itemize}
\end{strength}

\begin{strength}[Conceptual Clarity]
The approach provides conceptual insights:
\begin{itemize}
\item Makes RH a spectral positivity statement
\item Connects number theory to mathematical physics
\item Suggests computational approaches to L-functions
\end{itemize}
\end{strength}

\subsection{Limitations and Criticisms}

\begin{limitation}[Technical Complexity]
The theory requires:
\begin{itemize}
\item Mastery of multiple advanced mathematical areas
\item Deep, non-obvious constructions at many steps
\item Verification of conditions that are extremely difficult to check
\end{itemize}
\end{limitation}

\begin{limitation}[Non-constructive Aspects]
Key elements of the theory:
\begin{itemize}
\item Are defined by existence theorems rather than explicit constructions
\item Often make explicit computations impossible
\item Create a significant gap between theory and computation
\end{itemize}
\end{limitation}

\begin{limitation}[Unresolved Issues]
Fundamental problems remain:
\begin{itemize}
\item The main gap identified by Conrey-Li remains open
\item No clear path exists for completing the proof
\item May require mathematical structures not yet conceived
\end{itemize}
\end{limitation}

\section{Related Developments and Modern Perspectives}
\label{sec:modern-perspectives}

\subsection{The Bombieri-Garrett Limitation}

Independent work by Bombieri and Garrett identified fundamental limitations in spectral approaches:

\begin{theorem}[Bombieri-Garrett Obstruction]
Regular spacing of spectral parameters conflicts with the statistical properties of zeta zeros, implying that at most a fraction of zeros can be realized as spectral parameters of any single operator.
\end{theorem}

This suggests that the de Branges approach, even if perfected, might not capture all zeta zeros through a single spectral realization.

\subsection{Connection to Random Matrix Theory}

Modern developments reveal deep connections between de Branges spaces and random matrix theory:

\begin{itemize}
\item \textbf{Zero statistics}: Match predictions from Gaussian Unitary Ensemble (GUE)
\item \textbf{Quantum chaos}: Suggests underlying quantum chaotic interpretation
\item \textbf{Spectral correlations}: Provide new perspectives on the zeta zero distribution
\end{itemize}

\subsection{Contemporary Applications}

Recent work has extended de Branges ideas to:

\begin{itemize}
\item \textbf{Quantum graphs}: Discrete analogues with explicit spectral realizations
\item \textbf{Non-commutative geometry}: Abstract framework for spectral approaches
\item \textbf{Arithmetic quantum chaos}: Connections between number theory and quantum mechanics
\end{itemize}

\section{Conclusion}
\label{sec:debranges-conclusion}

The de Branges theory represents one of the most ambitious and sophisticated approaches to the Riemann Hypothesis. While it has not succeeded in proving RH, its impact on mathematics extends far beyond this single problem.

\subsection{Lasting Contributions}

The theory has provided:

\begin{enumerate}
\item \textbf{Powerful mathematical tools} with applications throughout analysis and operator theory
\item \textbf{Deep structural insights} into the connections between different areas of mathematics  
\item \textbf{Conceptual framework} for understanding RH as a spectral problem
\item \textbf{Influence on modern approaches} to L-functions and automorphic forms
\end{enumerate}

\subsection{Current Challenges}

The main obstacles facing the approach are:

\begin{enumerate}
\item \textbf{Closing the Conrey-Li gap}: The central technical obstacle
\item \textbf{Making constructions explicit}: Moving beyond existence theorems
\item \textbf{Developing computational methods}: Bridging theory and computation  
\item \textbf{Understanding fundamental limitations}: Accepting what the approach cannot achieve
\end{enumerate}

\subsection{Future Prospects}

Whether de Branges theory can ultimately prove RH remains an open question. However, the theory has already:

\begin{itemize}
\item Enriched our understanding of the deep connections between analysis, operator theory, and number theory
\item Provided a framework that continues to generate new mathematics
\item Influenced the development of related approaches to fundamental problems
\item Demonstrated the power of unifying abstract and concrete mathematical perspectives
\end{itemize}

The de Branges approach stands as a testament to the depth and interconnectedness of mathematics, showing how the pursuit of one profound problem can illuminate vast regions of mathematical landscape, even when the original goal remains tantalizingly out of reach.

\begin{remark}[Final Assessment]
As of 2024, the de Branges approach to RH represents both a remarkable mathematical achievement and a cautionary tale about the limits of current mathematical techniques. While the theory has not delivered a proof of RH, it has created lasting mathematical structures and insights that continue to influence research in analysis, operator theory, and number theory. The approach remains an active area of investigation, with researchers continuing to explore whether fundamental new insights might yet overcome the identified obstacles.
\end{remark}