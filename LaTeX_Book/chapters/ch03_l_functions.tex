% Chapter title is in main.tex
% Label is in main.tex

\begin{quote}
\textit{``The Selberg class provides a unified framework for understanding the deep structural properties shared by all L-functions arising from number theory and automorphic representation theory.''} --- Atle Selberg, 1992 \cite{selberg1992}
\end{quote}

\section{Introduction}

The theory of L-functions represents one of the most profound and unifying themes in modern number theory. From Riemann's original investigation of $\zeta(s)$ to the vast landscape of automorphic L-functions, these analytic objects encode fundamental arithmetic information through their analytic properties. The Selberg class, introduced by Atle Selberg in 1989 and formalized in 1992, provides an axiomatic framework that captures the essential properties of L-functions while being general enough to potentially include new, undiscovered examples.

This chapter explores the rich theory of L-functions through the lens of the Selberg class, examining how four simple axioms unite diverse mathematical objects and reveal deep structural constraints. We will see how classification results have eliminated entire ranges of possible degrees, how forbidden conductors impose unexpected arithmetic constraints, and how Selberg's conjectures connect to fundamental problems in algebraic number theory.

\section{Dirichlet L-functions and Extensions}
\label{sec:dirichlet-l-functions}

\subsection{Classical Dirichlet L-functions}

The natural generalization of the Riemann zeta function leads us to Dirichlet L-functions, which play a fundamental role in the theory of primes in arithmetic progressions.

\begin{definition}[Dirichlet Character]
A \emph{Dirichlet character} modulo $q$ is a completely multiplicative function $\chi: \mathbb{Z} \to \mathbb{C}$ such that:
\begin{enumerate}
\item $\chi(n) = 0$ if $\gcd(n,q) > 1$
\item $\chi(n) = \chi(m)$ if $n \equiv m \pmod{q}$
\item $\chi(nm) = \chi(n)\chi(m)$ for all $n,m \in \mathbb{Z}$
\end{enumerate}
A character $\chi$ is \emph{primitive} if it is not induced by a character of smaller conductor.
\end{definition}

\begin{definition}[Dirichlet L-function]
For a Dirichlet character $\chi$ modulo $q$, the associated \emph{Dirichlet L-function} is defined by:
\begin{equation}
L(s,\chi) = \sum_{n=1}^{\infty} \frac{\chi(n)}{n^s}
\end{equation}
for $\Re(s) > 1$.
\end{definition}

\begin{theorem}[Analytic Properties of Dirichlet L-functions]
Let $\chi$ be a primitive character modulo $q$. Then:
\begin{enumerate}
\item $L(s,\chi)$ has analytic continuation to $\mathbb{C}$
\item If $\chi$ is non-principal, $L(s,\chi)$ is entire
\item If $\chi$ is principal, $L(s,\chi)$ has a simple pole at $s=1$
\item $L(s,\chi)$ satisfies the functional equation:
\begin{equation}
\Lambda(s,\chi) = \varepsilon(\chi) \Lambda(1-s,\bar{\chi})
\end{equation}
where $\Lambda(s,\chi) = \left(\frac{q}{\pi}\right)^{s/2} \Gamma\left(\frac{s+a}{2}\right) L(s,\chi)$, with $a = 0$ if $\chi(-1) = 1$ and $a = 1$ if $\chi(-1) = -1$
\end{enumerate}
\end{theorem}

\subsection{The Prime Number Theorem in Arithmetic Progressions}

The most celebrated application of Dirichlet L-functions is the proof of the infinitude of primes in arithmetic progressions.

\begin{theorem}[Dirichlet's Theorem]
For any integers $a$ and $q$ with $\gcd(a,q) = 1$, there are infinitely many primes $p \equiv a \pmod{q}$.

Moreover, the primes are equidistributed among the reduced residue classes:
\begin{equation}
\pi(x; q, a) \sim \frac{1}{\phi(q)} \frac{x}{\log x}
\end{equation}
as $x \to \infty$.
\end{theorem}

\begin{proof}[Proof Sketch]
The key insight is that:
\begin{equation}
\sum_{p \equiv a \pmod{q}} \frac{1}{p^s} = \frac{1}{\phi(q)} \sum_{\chi \bmod q} \bar{\chi}(a) \sum_p \frac{\chi(p)}{p^s}
\end{equation}

For the principal character, $\sum_p \frac{1}{p^s}$ diverges logarithmically as $s \to 1^+$. For non-principal characters, $L(1,\chi) \neq 0$ (a deep result), ensuring the sum converges. This proves infinitude and the asymptotic formula follows from more careful analysis.
\end{proof}

\begin{remark}
The non-vanishing $L(1,\chi) \neq 0$ for non-principal characters is equivalent to the statement that $\zeta(s)$ has no zeros on $\Re(s) = 1$, connecting Dirichlet's theorem directly to properties of the Riemann zeta function.
\end{remark}

\subsection{Hecke L-functions and Algebraic Number Fields}

The theory extends naturally to algebraic number fields through Hecke's generalization.

\begin{definition}[Hecke Character]
Let $K$ be a number field with ring of integers $\mathcal{O}_K$. A \emph{Hecke character} (or Größencharakter) is a multiplicative function $\chi$ on the group of fractional ideals of $K$ that factors through the narrow class group and satisfies certain conditions at infinite places.
\end{definition}

\begin{definition}[Hecke L-function]
For a Hecke character $\chi$, the associated \emph{Hecke L-function} is:
\begin{equation}
L(s,\chi) = \sum_{\mathfrak{a}} \frac{\chi(\mathfrak{a})}{N(\mathfrak{a})^s} = \prod_{\mathfrak{p}} \left(1 - \frac{\chi(\mathfrak{p})}{N(\mathfrak{p})^s}\right)^{-1}
\end{equation}
where the sum is over all integral ideals $\mathfrak{a}$ and the product is over all prime ideals $\mathfrak{p}$.
\end{definition}

These L-functions satisfy functional equations analogous to those of classical Dirichlet L-functions and play crucial roles in class field theory and the Langlands program.

\section{Tate's Thesis and the Adelic Viewpoint}
\label{sec:tate-thesis}

In 1950, John Tate's revolutionary doctoral thesis \cite{tate1950,cassels1967} recast the theory of L-functions in the language of harmonic analysis on adele groups, providing a unified framework that reveals the deep structural reasons behind functional equations and analytic properties of L-functions.

\subsection{The Adele Ring}

The key innovation is to work simultaneously with all completions of $\mathbb{Q}$ (or more generally, a global field).

\begin{definition}[Adele Ring]
The \emph{adele ring} of $\mathbb{Q}$, denoted $\mathbb{A}_{\mathbb{Q}}$ or simply $\mathbb{A}$, is the restricted product:
\begin{equation}
\mathbb{A} = \mathbb{R} \times \prod_{p \text{ prime}}' \mathbb{Q}_p
\end{equation}
where the restricted product means elements $(x_\infty, x_2, x_3, x_5, \ldots)$ with $x_p \in \mathbb{Z}_p$ for all but finitely many primes $p$.
\end{definition}

\begin{remark}[Local-Global Principle]
The adelic approach encodes both local information (at each prime $p$ and at infinity) and global information (from $\mathbb{Q}$) in a single object. The diagonal embedding $\mathbb{Q} \hookrightarrow \mathbb{A}$ given by $x \mapsto (x, x, x, \ldots)$ realizes $\mathbb{Q}$ as a discrete cocompact subgroup of $\mathbb{A}$.
\end{remark}

\subsection{The Idele Group and Characters}

\begin{definition}[Idele Group]
The \emph{idele group} $\mathbb{I} = \mathbb{A}^{\times}$ is the group of units of the adele ring, consisting of elements $(x_v)$ with $x_v \neq 0$ and $x_p \in \mathbb{Z}_p^{\times}$ for all but finitely many $p$.
\end{definition}

The idele group comes with a natural norm map:
\begin{equation}
|\cdot| : \mathbb{I} \to \mathbb{R}_+^{\times}, \quad |(x_v)| = |x_\infty|_\infty \prod_p |x_p|_p
\end{equation}

\begin{theorem}[Product Formula]
For any $x \in \mathbb{Q}^{\times} \subset \mathbb{I}$:
\begin{equation}
|x| = |x|_\infty \prod_p |x|_p = 1
\end{equation}
This fundamental identity encodes the compatibility between different valuations.
\end{theorem}

\subsection{Tate's Local Zeta Functions}

For each place $v$ (prime $p$ or $\infty$), Tate defined local zeta functions:

\begin{definition}[Local Zeta Function]
For a Schwartz-Bruhat function $f_v$ on $\mathbb{Q}_v$ and a character $\chi_v$ of $\mathbb{Q}_v^{\times}$:
\begin{equation}
Z_v(f_v, \chi_v, s) = \int_{\mathbb{Q}_v^{\times}} f_v(x) \chi_v(x) |x|_v^s d^{\times}x
\end{equation}
\end{definition}

\begin{theorem}[Local Functional Equation]
Each local zeta function satisfies a functional equation:
\begin{equation}
Z_v(\hat{f}_v, \chi_v^{-1}, 1-s) = \gamma_v(\chi_v, s) Z_v(f_v, \chi_v, s)
\end{equation}
where $\hat{f}_v$ is the Fourier transform and $\gamma_v$ is a local gamma factor.
\end{theorem}

\subsection{Global Adelic Zeta Functions}

The global theory emerges from taking products over all places:

\begin{definition}[Global Zeta Function]
For $f = \prod_v f_v$ a Schwartz-Bruhat function on $\mathbb{A}$ and $\chi = \prod_v \chi_v$ a character of $\mathbb{I}$:
\begin{equation}
Z(f, \chi, s) = \int_{\mathbb{I}} f(x) \chi(x) |x|^s d^{\times}x
\end{equation}
\end{definition}

\begin{theorem}[Tate's Main Theorem]
The global zeta function can be written as:
\begin{equation}
Z(f, \chi, s) = \prod_v Z_v(f_v, \chi_v, s)
\end{equation}
and satisfies the functional equation:
\begin{equation}
Z(\hat{f}, \chi^{-1}, 1-s) = Z(f, \chi, s)
\end{equation}
\end{theorem}

\subsection{Recovery of Classical L-functions}

Tate showed how classical L-functions emerge from specific choices of test functions:

\begin{theorem}[Riemann Zeta via Tate's Method]
Taking $f = \prod_v f_v$ where:
\begin{itemize}
\item $f_\infty(x) = e^{-\pi x^2}$ for $x \in \mathbb{R}$
\item $f_p = \mathbf{1}_{\mathbb{Z}_p}$ (characteristic function) for all primes $p$
\end{itemize}
and $\chi$ the trivial character, we obtain:
\begin{equation}
Z(f, \chi, s) = \pi^{-s/2} \Gamma(s/2) \zeta(s)
\end{equation}
\end{theorem}

\begin{proof}[Proof Sketch]
The local computations give:
\begin{align}
Z_\infty(f_\infty, 1, s) &= \int_{\mathbb{R}^{\times}} e^{-\pi x^2} |x|^s dx = \pi^{-s/2} \Gamma(s/2) \\
Z_p(f_p, 1, s) &= \int_{\mathbb{Z}_p^{\times}} |x|_p^s dx + \int_{p\mathbb{Z}_p \setminus \{0\}} |x|_p^s dx = \frac{1}{1 - p^{-s}}
\end{align}
The product formula then yields the completed Riemann zeta function.
\end{proof}

\subsection{The Adelic Poisson Summation Formula}

The functional equation emerges naturally from the Poisson summation formula on $\mathbb{A}/\mathbb{Q}$:

\begin{theorem}[Adelic Poisson Summation]
For $f$ a Schwartz-Bruhat function on $\mathbb{A}$:
\begin{equation}
\sum_{\alpha \in \mathbb{Q}} f(\alpha) = \sum_{\alpha \in \mathbb{Q}} \hat{f}(\alpha)
\end{equation}
where $\hat{f}$ is the adelic Fourier transform.
\end{theorem}

This single formula encodes:
\begin{itemize}
\item Classical Poisson summation when restricted to $\mathbb{R}$
\item $p$-adic Poisson summation at each prime
\item The functional equations of all L-functions
\end{itemize}

\subsection{Advantages of the Adelic Approach}

The adelic viewpoint provides several crucial insights:

\begin{enumerate}
\item \textbf{Unified Treatment}: All L-functions (Riemann, Dirichlet, Hecke) emerge from the same framework by varying the character $\chi$

\item \textbf{Natural Functional Equation}: The functional equation becomes a consequence of Fourier duality rather than a mysterious identity

\item \textbf{Local-Global Principle}: The factorization $L(s) = \prod_v L_v(s)$ reflects the product structure of the adeles

\item \textbf{Conceptual Clarity}: The appearance of gamma factors, conductors, and epsilon factors becomes natural from the local theory

\item \textbf{Generalization}: The framework extends naturally to:
   \begin{itemize}
   \item GL(n) and automorphic L-functions
   \item Function fields (replacing $\mathbb{Q}$ with $\mathbb{F}_q(t)$)
   \item The geometric Langlands program
   \end{itemize}
\end{enumerate}

\subsection{Connection to the Riemann Hypothesis}

While Tate's thesis doesn't directly prove RH, it provides structural insights:

\begin{theorem}[Weil's Explicit Formula via Adeles]
The explicit formula for prime distribution can be reformulated adelically as:
\begin{equation}
\sum_p \log p \cdot f(\log p) = -\sum_{\rho} \hat{f}(\rho) + \text{boundary terms}
\end{equation}
where the sum is over zeros $\rho$ of the zeta function.
\end{theorem}

\begin{remark}[RH as Spectral Problem]
In the adelic framework, RH becomes equivalent to certain positivity conditions on distributions on $\mathbb{I}/\mathbb{Q}^{\times}$. This connects to:
\begin{itemize}
\item Weil's positivity criterion
\item Connes' trace formula approach
\item The spectral interpretation of zeros
\end{itemize}
\end{remark}

\subsection{Modern Developments}

The adelic approach has become fundamental to modern number theory:

\begin{enumerate}
\item \textbf{Langlands Program}: Automorphic representations on adelic groups generalize classical modular forms

\item \textbf{Iwasawa Theory}: $p$-adic L-functions arise naturally from the $p$-adic components of adelic L-functions

\item \textbf{Arithmetic Geometry}: The adelic viewpoint connects to:
   \begin{itemize}
   \item Height pairings on algebraic varieties
   \item Arakelov theory
   \item The Tamagawa number conjecture
   \end{itemize}

\item \textbf{Physics Connections}: The adelic formulation appears in:
   \begin{itemize}
   \item $p$-adic string theory
   \item Quantum field theory on adelic spaces
   \item Statistical mechanics models
   \end{itemize}
\end{enumerate}

\begin{remark}[Historical Impact]
Tate's thesis fundamentally changed how mathematicians think about L-functions. What appeared as miraculous identities in classical theory became natural consequences of harmonic analysis on locally compact groups. This conceptual clarity has been essential for all subsequent developments in the theory of L-functions and automorphic forms.
\end{remark}

\subsection{The Adelic Riemann Hypothesis}

The RH can be reformulated entirely in adelic terms:

\begin{conjecture}[Adelic RH]
Let $\chi$ be a non-trivial character of $\mathbb{I}/\mathbb{Q}^{\times}$ with $|\chi| = |\cdot|^{i\gamma}$. Then the global zeta function $Z(f, \chi, s)$ has all its zeros on the line $\Re(s) = 1/2$.
\end{conjecture}

This formulation:
\begin{itemize}
\item Treats all places symmetrically
\item Suggests RH is about harmonic analysis on $\mathbb{I}/\mathbb{Q}^{\times}$
\item Connects to representation theory of reductive groups
\item Provides a framework for generalizations to other L-functions
\end{itemize}

The adelic perspective thus transforms RH from a problem about a specific function to a question about symmetries and harmonic analysis on fundamental arithmetic objects.

\section{The Selberg Class Framework}
\label{sec:selberg-class}

\subsection{Axiomatic Definition}

Selberg's profound insight was that L-functions share certain fundamental properties that can be axiomatized, allowing for a unified treatment.

\begin{definition}[The Selberg Class $\mathcal{S}$]
A Dirichlet series $F(s) = \sum_{n=1}^{\infty} \frac{a(n)}{n^s}$ belongs to the \emph{Selberg class} $\mathcal{S}$ if it satisfies four axioms:

\textbf{Axiom 1 (Dirichlet Series):} $F(s)$ is absolutely convergent for $\Re(s) > 1$.

\textbf{Axiom 2 (Analytic Continuation):} There exists an integer $m \geq 0$ such that $(s-1)^m F(s)$ is an entire function of finite order.

\textbf{Axiom 3 (Functional Equation):} $F$ satisfies a functional equation of the form:
\begin{equation}
\Phi(s) = \omega \overline{\Phi(1-\bar{s})}
\end{equation}
where:
\begin{align}
\Phi(s) &= Q^s \prod_{j=1}^r \Gamma(\lambda_j s + \mu_j) F(s) \\
Q &> 0 \text{ (the conductor)} \\
\lambda_j &> 0 \\
\Re(\mu_j) &\geq 0 \\
|\omega| &= 1
\end{align}

\textbf{Axiom 4 (Euler Product):} $F$ has an Euler product of the form:
\begin{equation}
F(s) = \prod_p \exp\left(\sum_{k=1}^{\infty} \frac{b(p^k)}{p^{ks}}\right)
\end{equation}
where:
\begin{itemize}
\item $b(p^k) \ll p^{k\theta}$ for some $\theta < 1/2$
\item $b(n) = 0$ unless $n$ is a prime power
\end{itemize}
\end{definition}

\begin{remark}[Critical Constraint]
The condition $\theta < 1/2$ in Axiom 4 is essential. Without it, the class would include functions that violate the Riemann Hypothesis, such as functions with zeros to the right of the critical line.
\end{remark}

\subsection{The Degree of an L-function}

\begin{definition}[Degree]
The \emph{degree} of $F \in \mathcal{S}$ is defined as:
\begin{equation}
d_F = 2\sum_{j=1}^r \lambda_j
\end{equation}
This is a fundamental invariant that measures the ``complexity'' of the L-function.
\end{definition}

\begin{theorem}[Uniqueness of Gamma Factors]
If $\Phi^{(1)}(s)$ and $\Phi^{(2)}(s)$ are both admissible gamma factors for $F$, then $\Phi^{(1)}(s) = C\Phi^{(2)}(s)$ for some positive constant $C$.
\end{theorem}

This theorem ensures that the degree $d_F$ is well-defined and independent of the choice of gamma factors.

\subsection{Classical Examples}

\begin{example}[The Riemann Zeta Function]
$\zeta(s) \in \mathcal{S}$ with:
\begin{itemize}
\item Degree: $d_\zeta = 1$
\item Conductor: $Q = 1$
\item Gamma factor: $\pi^{-s/2}\Gamma(s/2)$
\item Functional equation: $\xi(s) = \xi(1-s)$ where $\xi(s) = \frac{1}{2}s(s-1)\pi^{-s/2}\Gamma(s/2)\zeta(s)$
\end{itemize}
\end{example}

\begin{example}[Dirichlet L-functions]
For a primitive character $\chi$ modulo $q$, $L(s,\chi) \in \mathcal{S}$ with:
\begin{itemize}
\item Degree: $d_{L(\cdot,\chi)} = 1$
\item Conductor: $Q = q$
\item The functional equation involves $\Gamma(s/2)$ or $\Gamma((s+1)/2)$ depending on the parity of $\chi$
\end{itemize}
\end{example}

\begin{example}[Dedekind Zeta Functions]
For a number field $K$, $\zeta_K(s) \in \mathcal{S}$ with:
\begin{itemize}
\item Degree: $d_{\zeta_K} = [K:\mathbb{Q}]$
\item Conductor: $Q = |\text{disc}(K)|$
\item The gamma factor involves $r_1$ copies of $\Gamma(s/2)$ and $r_2$ copies of $\Gamma(s)$ where $r_1$ and $r_2$ are the numbers of real and complex embeddings
\end{itemize}
\end{example}

\subsection{The Extended Selberg Class}

\begin{definition}[Extended Selberg Class $\mathcal{S}^{\sharp}$]
The \emph{extended Selberg class} $\mathcal{S}^{\sharp}$ consists of functions satisfying Axioms 1, 2, and 3 but not necessarily Axiom 4 (the Euler product condition).
\end{definition}

The extended class $\mathcal{S}^{\sharp}$ is technically useful because:
\begin{itemize}
\item It is closed under multiplication
\item It admits unique factorization into primitive functions
\item It is more amenable to analytical techniques
\item Results about $\mathcal{S}^{\sharp}$ often lead to results about $\mathcal{S}$ itself
\end{itemize}

\section{The Degree Conjecture}
\label{sec:degree-conjecture}

\subsection{Statement and Significance}

\begin{conjecture}[The Degree Conjecture]
For every $F \in \mathcal{S}$, the degree $d_F$ is a non-negative integer.
\end{conjecture}

This conjecture is central to the theory of the Selberg class. It asserts that the continuous parameter $d_F = 2\sum_{j=1}^r \lambda_j$ must actually take only discrete values, revealing a fundamental quantization in the structure of L-functions.

\subsection{Classification Results}

The degree conjecture has been proven for small degrees through a series of remarkable results:

\begin{theorem}[Complete Classification for Small Degrees]
The following results have been established:

\textbf{Degree 0:} $\mathcal{S}_0 = \{1\}$ \cite{conreyghosh1993}

\textbf{Degrees $0 < d < 1$:} $\mathcal{S}_d = \emptyset$ \cite{conreyghosh1993}

\textbf{Degree 1:} Complete classification \cite{kaczorowskiperelli1999}
$\mathcal{S}_1$ consists exactly of $\zeta(s)$ and shifted Dirichlet L-functions $L(s+i\tau, \chi)$

\textbf{Degrees $1 < d < 2$:} $\mathcal{S}_d = \emptyset$ \cite{kaczorowskiperelli2020}

\textbf{All degrees $d < 5/3$:} The degree conjecture is proven \cite{kaczorowskiperelli2020}
\end{theorem}

\begin{proof}[Proof Strategy for $1 < d < 2$]
The proof uses sophisticated techniques involving nonlinear twists:

1. \textbf{Nonlinear Twists:} For $F \in \mathcal{S}$ with degree $d$, consider:
   \begin{equation}
   F_d(s,\alpha) = \sum_{n=1}^{\infty} \frac{a_F(n)}{n^s} e(-n^{1/d}\alpha)
   \end{equation}

2. \textbf{Transformation Properties:} These twists satisfy functional equations that impose constraints on the coefficient structure.

3. \textbf{Spectral Analysis:} The poles of $F_d(s,\alpha)$ are determined by the spectrum $\text{Spec}(F) = \{\alpha > 0 : a_F(n_\alpha) \neq 0\}$.

4. \textbf{The Contradiction:} For $1 < d < 2$, the transformation properties force the existence of a linear sequence in the support of $a_F$, which would imply $d = 1$, a contradiction.
\end{proof}

\subsection{Degree 2 Classification}

For degree 2 functions, a complete classification has been achieved in special cases:

\begin{theorem}[Degree 2, Conductor 1 Classification]
Every $F \in \mathcal{S}^{\sharp}$ with degree 2 and conductor 1 is one of:
\begin{enumerate}
\item $\zeta(s)^2$
\item L-function of a holomorphic cusp form of weight $k \geq 12$ and level 1
\item L-function of a Maass cusp form of level 1
\end{enumerate}

The classification is determined by the eigenweight invariant:
\begin{equation}
\chi_F = \xi_F + H_F(2) + \frac{2}{3}
\end{equation}
where:
\begin{itemize}
\item $\chi_F > 0$: holomorphic cusp forms
\item $\chi_F = 0$: $\zeta(s)^2$
\item $\chi_F < 0$: Maass forms
\end{itemize}
\end{theorem}

This result provides strong evidence that all functions in $\mathcal{S}$ arise from automorphic representations, supporting the fundamental conjecture that $\mathcal{S}$ consists exactly of automorphic L-functions.

\section{Automorphic L-functions}
\label{sec:automorphic-l-functions}

\subsection{Modular Forms and Their L-functions}

The connection between modular forms and L-functions provides a rich source of examples in the Selberg class.

\begin{definition}[Modular Form]
A \emph{modular form} of weight $k$ and level $N$ is a holomorphic function $f$ on the upper half-plane $\mathbb{H}$ satisfying:
\begin{enumerate}
\item $f\left(\frac{az+b}{cz+d}\right) = (cz+d)^k f(z)$ for all $\begin{pmatrix} a & b \\ c & d \end{pmatrix} \in \Gamma_0(N)$
\item $f$ is holomorphic at the cusps
\item If $k = 0$, then $f$ has mean value zero on $\Gamma_0(N)\backslash\mathbb{H}$
\end{enumerate}
\end{definition}

\begin{definition}[L-function of a Modular Form]
For a normalized eigenform $f(z) = \sum_{n=1}^{\infty} a_n e^{2\pi i nz}$ of weight $k$ and level $N$, the associated L-function is:
\begin{equation}
L(s,f) = \sum_{n=1}^{\infty} \frac{a_n}{n^s}
\end{equation}
\end{definition}

\begin{theorem}[Properties of Modular L-functions]
If $f$ is a newform of weight $k$ and level $N$, then $L(s,f) \in \mathcal{S}$ with:
\begin{itemize}
\item Degree: $d_f = 2$
\item Conductor: $Q = N$
\item Functional equation involving $\Gamma(s + (k-1)/2)$
\item The coefficients $a_n$ satisfy the Ramanujan conjecture: $|a_p| \leq 2p^{(k-1)/2}$ \cite{iwanieckowalski2004}
\end{itemize}
\end{theorem}

\subsection{Maass Forms and Spectral Theory}

\begin{definition}[Maass Form]
A \emph{Maass form} is a smooth function $u$ on $\Gamma\backslash\mathbb{H}$ that is:
\begin{enumerate}
\item An eigenfunction of the Laplacian: $\Delta u = \lambda u$
\item Of moderate growth at the cusps
\item Orthogonal to constants (if $\lambda = 0$)
\end{enumerate}
where $\Delta = -y^2\left(\frac{\partial^2}{\partial x^2} + \frac{\partial^2}{\partial y^2}\right)$ is the hyperbolic Laplacian.
\end{definition}

For a Maass form $u$ with eigenvalue $\lambda = s(1-s)$ where $s = 1/2 + ir$ with $r \in \mathbb{R}$, the associated L-function has degree 2 and functional equation involving $\Gamma(s \pm ir)$.

\subsection{The Langlands Program Perspective}

The Langlands program provides a conceptual framework for understanding all automorphic L-functions:

\begin{conjecture}[Langlands Reciprocity]
There is a bijective correspondence between:
\begin{itemize}
\item $n$-dimensional irreducible representations of $\text{Gal}(\overline{\mathbb{Q}}/\mathbb{Q})$
\item Cuspidal automorphic representations of $\text{GL}_n(\mathbb{A}_\mathbb{Q})$
\end{itemize}
that preserves L-functions.
\end{conjecture}

This conjecture, if true, would imply that all ``motivic'' L-functions (arising from algebraic geometry) are automorphic, and hence belong to the Selberg class.

\begin{theorem}[Automorphic L-functions in $\mathcal{S}$]
If $\pi$ is a cuspidal automorphic representation of $\text{GL}_n(\mathbb{A}_\mathbb{Q})$ satisfying the Ramanujan conjecture, then $L(s,\pi) \in \mathcal{S}$ with degree $n$.
\end{theorem}

The converse question---whether every element of $\mathcal{S}$ arises from an automorphic representation---is a fundamental open problem.

\section{Forbidden Conductors and Arithmetic Constraints}
\label{sec:forbidden-conductors}

\subsection{The Discovery of Forbidden Conductors}

One of the most surprising recent discoveries in the theory of the Selberg class is that not all positive real numbers can serve as conductors of L-functions.

\begin{theorem}[Existence of Forbidden Conductors]
Not all positive real numbers can be conductors of degree-2 L-functions in $\mathcal{S}^{\sharp}$.

Specifically, a positive integer $q > 1$ is a \emph{forbidden conductor} if:
\begin{enumerate}
\item All prime divisors $p$ of $q$ satisfy $p \equiv 3 \pmod{4}$
\item The Jacobi symbol $(2|q) = -1$
\item The continued fraction of $\sqrt{q}$ has period length 1
\end{enumerate}
\end{theorem}

\begin{example}[Forbidden Integer Conductors]
The following integers are forbidden conductors:
\begin{equation}
3, 7, 11, 19, 23, 31, 43, 47, 59, 67, 71, 79, 83, 103, \ldots
\end{equation}
\end{example}

\subsection{The Mathematical Mechanism}

The obstruction arises through a deep connection to continued fractions:

\begin{definition}[Weight Function]
For a conductor $q > 0$ and a vector $\mathbf{m} = (m_0, m_1, \ldots, m_k) \in \mathbb{Z}^{k+1}$, define:
\begin{equation}
c(q, \mathbf{m}) = m_k + \frac{1}{qm_{k-1} + \frac{q}{qm_{k-2} + \frac{q}{\ddots + \frac{q}{qm_0}}}}
\end{equation}

The \emph{weight function} is:
\begin{equation}
w(q, \mathbf{m}) = q^{k/2} \prod_{j=0}^{k-1} |c(q, \mathbf{m}_j)|
\end{equation}
where $\mathbf{m}_j = (m_0, \ldots, m_j)$.
\end{definition}

\begin{theorem}[Fundamental Criterion for Forbidden Conductors]
A conductor $q$ is forbidden if there exists a proper loop $\mathbf{m}$ (i.e., $c(q, \mathbf{m}) = 0$ with all $m_j \neq 0$ for $j = 0, \ldots, k-1$) such that $w(q, \mathbf{m}) \neq 1$.
\end{theorem}

\subsection{Density and Distribution Results}

\begin{theorem}[Density of Forbidden Conductors]
The set of forbidden conductors is dense in the interval $(0, 4)$.
\end{theorem}

This remarkable result shows that forbidden conductors are not isolated exceptions but form a dense subset, revealing unexpected arithmetic constraints on the structure of L-functions.

\begin{theorem}[Explicit Forbidden Families]
The following are forbidden conductors:
\begin{equation}
q = \frac{4}{n} \cos^2\left(\frac{\pi\ell}{2k+1}\right)
\end{equation}
where $k \geq 1$, $1 \leq \ell < 2k+1$, $\gcd(\ell, 2k+1) = 1$, and $n \geq 2$.
\end{theorem}

The set of rational forbidden conductors has accumulation points at $(3-\sqrt{5})/2 \approx 0.382$ and $(3+\sqrt{5})/2 \approx 2.618$, values related to the golden ratio.

\subsection{Implications for the Selberg Class}

The discovery of forbidden conductors has several profound implications:

\begin{enumerate}
\item \textbf{Structural Rigidity:} The Selberg class has more rigid structure than initially expected, with arithmetic constraints limiting which analytic structures can be realized.

\item \textbf{Computational Tools:} The theory provides explicit algorithms to test whether a given real number can serve as a conductor.

\item \textbf{Connection to Diophantine Theory:} The continued fraction approach reveals unexpected connections between L-functions and classical Diophantine analysis.

\item \textbf{RH Implications:} Any approach to proving the Riemann Hypothesis for the Selberg class must account for these forbidden conductor constraints.
\end{enumerate}

\section{Selberg's Conjectures and Implications}
\label{sec:selberg-conjectures}

\subsection{The Fundamental Conjectures}

Selberg proposed several deep conjectures about the multiplicative structure of the class $\mathcal{S}$:

\begin{conjecture}[Conjecture A: Primitivity]
For $F \in \mathcal{S}$ primitive, there exists a positive integer $n_F$ such that:
\begin{equation}
\sum_{p \leq x} \frac{|a_F(p)|^2}{p} = n_F \log \log x + O(1)
\end{equation}
\end{conjecture}

\begin{conjecture}[Conjecture B: Orthogonality]
For distinct primitive functions $F, G \in \mathcal{S}$:
\begin{equation}
\sum_{p \leq x} \frac{a_F(p)\overline{a_G(p)}}{p} = O(1)
\end{equation}
\end{conjecture}

These conjectures encode a fundamental orthogonality principle: primitive L-functions are orthogonal in a precise quantitative sense.

\subsection{Unique Factorization}

\begin{theorem}[Factorization Properties]
\begin{enumerate}
\item Every function $F \in \mathcal{S}$ can be written as a product of primitive functions.
\item Conjecture B implies that this factorization is unique.
\item If $F = F_1^{e_1} \cdots F_k^{e_k}$ where the $F_i$ are distinct primitives, then $n_F = e_1^2 + \cdots + e_k^2$.
\end{enumerate}
\end{theorem}

\begin{proof}[Proof of Unique Factorization]
Suppose $F = \prod F_i^{e_i} = \prod G_j^{f_j}$ are two primitive factorizations. Taking logarithms and using orthogonality:
\begin{align}
\sum_{p \leq x} \frac{|a_F(p)|^2}{p} &= \sum_{p \leq x} \frac{\left|\prod a_{F_i}(p)^{e_i}\right|^2}{p} \\
&= \sum_{i,j} e_i \overline{e_j} \sum_{p \leq x} \frac{a_{F_i}(p)\overline{a_{F_j}(p)}}{p} \\
&= \sum_i |e_i|^2 \log \log x + O(1)
\end{align}
by Conjecture B. Similarly for the other factorization, giving uniqueness.
\end{proof}

\subsection{Connection to the Artin Conjecture}

\begin{conjecture}[Artin Conjecture]
If $\rho: \text{Gal}(\overline{\mathbb{Q}}/\mathbb{Q}) \to \text{GL}_n(\mathbb{C})$ is an irreducible representation with $n > 1$, then $L(s,\rho)$ is entire.
\end{conjecture}

\begin{theorem}[Murty's Result \cite{iwanieckowalski2004}]
Selberg's Conjecture B implies the Artin conjecture.
\end{theorem}

\begin{proof}[Proof Sketch]
\begin{enumerate}
\item Use Brauer induction to write $L(s,\rho) = L(s,\chi_1)/L(s,\chi_2)$ where $\chi_1, \chi_2$ are products of Hecke L-functions \cite{langlands1976}.

\item Both numerator and denominator belong to $\mathcal{S}$ with primitive factorizations.

\item Write $L(s,\rho) = \prod F_i(s)^{e_i}$ where $F_i$ are primitive and $e_i \in \mathbb{Z}$.

\item Use the Chebotarev density theorem:
   \begin{equation}
   \sum_{p \leq x} \frac{|\rho(\text{Frob}_p)|^2}{p} = \log \log x + O(1)
   \end{equation}

\item By Conjecture B orthogonality: $\sum_i e_i^2 = 1$.

\item Since $e_i \in \mathbb{Z}$, we must have exactly one $e_i = \pm 1$ and the rest zero.

\item Therefore $L(s,\rho) = F(s)$ or $1/F(s)$ for some primitive $F$.

\item Since $L(s,\rho)$ has no poles for irreducible $\rho \neq 1$, we get $L(s,\rho) = F(s)$ primitive and entire.
\end{enumerate}
\end{proof}

\subsection{Connection to the Langlands Program}

\begin{theorem}[Solvable Case of Langlands Reciprocity]
Assume Conjecture B. Let $K/\mathbb{Q}$ be a Galois extension with solvable group $G$, and let $\chi$ be an irreducible character of degree $n$. Then there exists an irreducible cuspidal automorphic representation $\pi$ of $\text{GL}_n(\mathbb{A}_\mathbb{Q})$ such that $L(s,\chi) = L(s,\pi)$ \cite{langlands1976}.
\end{theorem}

This result shows how Selberg's analytic conjectures provide an alternative pathway to fundamental results in the Langlands program, at least for solvable Galois groups.

\subsection{Universality and Independence}

\begin{theorem}[Joint Universality]
Assume Selberg's orthogonality conjecture. Then distinct primitive functions in $\mathcal{S}$ are jointly universal---they simultaneously approximate arbitrary analytic functions in appropriate regions of the critical strip.
\end{theorem}

This provides a quantitative version of the independence of L-functions and has applications to value distribution theory and simultaneous non-vanishing problems.

\section{Open Problems and Future Directions}

\subsection{Major Conjectures}

\begin{enumerate}
\item \textbf{Complete Degree Conjecture:} Prove that $d_F \in \mathbb{N}$ for all $F \in \mathcal{S}$.

\item \textbf{Automorphic Characterization:} Prove that $\mathcal{S}$ consists exactly of automorphic L-functions.

\item \textbf{Selberg's Orthogonality:} Prove Conjecture B and its implications for unique factorization.

\item \textbf{Higher Degree Classification:} Extend classification results beyond degree 2.

\item \textbf{Forbidden Conductors:} Characterize all forbidden conductors and extend the theory to higher degrees.
\end{enumerate}

\subsection{Connections to the Riemann Hypothesis}

The Selberg class framework provides several potential approaches to RH:

\begin{enumerate}
\item \textbf{Unified Proof:} Any proof of RH for all functions in $\mathcal{S}$ would immediately apply to all classical L-functions.

\item \textbf{Structural Constraints:} Classification results and forbidden conductors impose structural constraints that any counterexample to RH would have to satisfy.

\item \textbf{Orthogonality Methods:} Selberg's conjectures suggest proof strategies based on orthogonality and independence of L-functions.

\item \textbf{Degree-Based Approach:} The complete understanding of small degrees might extend to general degree bounds for zeros.
\end{enumerate}

\subsection{Computational Aspects}

\begin{enumerate}
\item \textbf{Testing Membership:} Develop algorithms to determine whether a given Dirichlet series belongs to $\mathcal{S}$.

\item \textbf{Classification Algorithms:} Systematic methods for classifying functions of given degree and conductor.

\item \textbf{Conductor Testing:} Efficient algorithms to determine if a real number is a forbidden conductor.

\item \textbf{Verification of Conjectures:} Numerical verification of Selberg's conjectures for specific families of L-functions.
\end{enumerate}

\section{Conclusion}

The Selberg class represents a fundamental organizing principle for L-function theory, providing both a conceptual framework and concrete results about the structure of these essential objects in number theory. The axiomatic approach has revealed unexpected constraints: entire degree ranges are impossible, conductors can be forbidden by arithmetic obstructions, and the multiplicative structure is governed by deep orthogonality principles.

The classification results for small degrees demonstrate the power of this framework, while the discovery of forbidden conductors shows that the interplay between analytic and arithmetic properties is more subtle than initially imagined. Selberg's conjectures connect this analytic theory to fundamental problems in algebraic number theory and the Langlands program, providing multiple pathways between different areas of mathematics.

The theory has already yielded profound insights into the structure of L-functions and continues to guide research toward a complete understanding of these objects. Whether through the degree conjecture, orthogonality relations, or geometric interpretations of structural constraints, the Selberg class framework ensures that future advances will apply broadly to all L-functions of arithmetic significance.

The ultimate goal---a complete characterization of all L-functions and a proof of their analytic properties including the Riemann Hypothesis---remains tantalizingly within reach. The Selberg class provides the natural setting for this grand synthesis, unifying centuries of research into a single, powerful theory that continues to reveal new mathematical truths about the deepest structures in number theory.