% Appendix A: Mathematical Prerequisites

This appendix provides a concise review of the mathematical background required for reading this book. We assume familiarity with undergraduate mathematics but provide key definitions and results for reference.

\section{Complex Analysis}

\subsection{Basic Concepts}

\begin{definition}[Holomorphic Function]
A function $f: U \to \mathbb{C}$ defined on an open set $U \subseteq \mathbb{C}$ is \emph{holomorphic} (or \emph{analytic}) at $z_0 \in U$ if the derivative
\[
f'(z_0) = \lim_{h \to 0} \frac{f(z_0 + h) - f(z_0)}{h}
\]
exists. The function is holomorphic on $U$ if it is holomorphic at every point of $U$.
\end{definition}

\begin{theorem}[Cauchy-Riemann Equations]
Let $f(z) = u(x,y) + iv(x,y)$ where $z = x + iy$. Then $f$ is holomorphic if and only if $u$ and $v$ have continuous partial derivatives satisfying:
\[
\frac{\partial u}{\partial x} = \frac{\partial v}{\partial y}, \quad \frac{\partial u}{\partial y} = -\frac{\partial v}{\partial x}
\]
\end{theorem}

\begin{theorem}[Cauchy's Integral Formula]
If $f$ is holomorphic inside and on a simple closed contour $\gamma$, then for any $z_0$ inside $\gamma$:
\[
f(z_0) = \frac{1}{2\pi i} \oint_{\gamma} \frac{f(z)}{z - z_0} dz
\]
\end{theorem}

\subsection{Laurent Series and Residues}

\begin{definition}[Laurent Series]
A function $f$ holomorphic in an annulus $r < |z - z_0| < R$ has a unique Laurent expansion:
\[
f(z) = \sum_{n=-\infty}^{\infty} a_n (z - z_0)^n
\]
where $a_n = \frac{1}{2\pi i} \oint_{\gamma} \frac{f(w)}{(w - z_0)^{n+1}} dw$ for any circle $\gamma$ in the annulus.
\end{definition}

\begin{definition}[Residue]
The \emph{residue} of $f$ at an isolated singularity $z_0$ is $\res_{z=z_0} f(z) = a_{-1}$ in the Laurent expansion.
\end{definition}

\begin{theorem}[Residue Theorem]
If $f$ is holomorphic inside and on a simple closed contour $\gamma$ except for isolated singularities $z_1, \ldots, z_n$ inside $\gamma$, then:
\[
\oint_{\gamma} f(z) dz = 2\pi i \sum_{k=1}^n \res_{z=z_k} f(z)
\]
\end{theorem}

\subsection{Entire and Meromorphic Functions}

\begin{definition}[Entire Function]
A function holomorphic on all of $\mathbb{C}$ is called \emph{entire}.
\end{definition}

\begin{definition}[Meromorphic Function]
A function that is holomorphic except for isolated poles is called \emph{meromorphic}.
\end{definition}

\begin{theorem}[Hadamard Factorization]
Any entire function $f$ of finite order $\rho$ can be written as:
\[
f(z) = e^{g(z)} z^m \prod_{n=1}^{\infty} E(z/z_n, p)
\]
where $g$ is a polynomial of degree at most $\rho$, $m$ is the order of the zero at the origin, $\{z_n\}$ are the non-zero zeros, and $E(z,p)$ is the Weierstrass primary factor.
\end{theorem}

\section{Number Theory}

\subsection{Arithmetic Functions}

\begin{definition}[Multiplicative Function]
A function $f: \mathbb{N} \to \mathbb{C}$ is \emph{multiplicative} if $f(1) = 1$ and $f(mn) = f(m)f(n)$ whenever $\gcd(m,n) = 1$. It is \emph{completely multiplicative} if $f(mn) = f(m)f(n)$ for all $m,n$.
\end{definition}

Important arithmetic functions:
\begin{itemize}
\item Euler's totient function: $\phi(n) = \#\{k \leq n : \gcd(k,n) = 1\}$
\item Möbius function: $\mu(n) = \begin{cases}
1 & \text{if } n = 1 \\
(-1)^k & \text{if } n = p_1 \cdots p_k \text{ (distinct primes)} \\
0 & \text{if } n \text{ has a squared prime factor}
\end{cases}$
\item Divisor function: $d(n) = \sum_{d|n} 1$
\item Sum of divisors: $\sigma(n) = \sum_{d|n} d$
\item von Mangoldt function: $\Lambda(n) = \begin{cases}
\log p & \text{if } n = p^k \\
0 & \text{otherwise}
\end{cases}$
\end{itemize}

\begin{theorem}[Möbius Inversion]
If $f(n) = \sum_{d|n} g(d)$, then $g(n) = \sum_{d|n} \mu(n/d) f(d)$.
\end{theorem}

\subsection{Dirichlet Characters}

\begin{definition}[Dirichlet Character]
A \emph{Dirichlet character} modulo $q$ is a completely multiplicative function $\chi: \mathbb{Z} \to \mathbb{C}$ such that:
\begin{enumerate}
\item $\chi(n) = 0$ if $\gcd(n,q) > 1$
\item $\chi(n) = \chi(n')$ if $n \equiv n' \pmod{q}$
\item $|\chi(n)| = 1$ if $\gcd(n,q) = 1$
\end{enumerate}
\end{definition}

\begin{theorem}[Orthogonality Relations]
For Dirichlet characters modulo $q$:
\[
\frac{1}{\phi(q)} \sum_{\chi} \chi(m) \overline{\chi}(n) = \begin{cases}
1 & \text{if } m \equiv n \pmod{q} \\
0 & \text{otherwise}
\end{cases}
\]
\end{theorem}

\section{Functional Analysis}

\subsection{Hilbert Spaces}

\begin{definition}[Inner Product Space]
A complex vector space $V$ with an inner product $\langle \cdot, \cdot \rangle: V \times V \to \mathbb{C}$ satisfying:
\begin{enumerate}
\item $\langle x, x \rangle \geq 0$ with equality iff $x = 0$
\item $\langle x, y \rangle = \overline{\langle y, x \rangle}$
\item $\langle ax + by, z \rangle = a\langle x, z \rangle + b\langle y, z \rangle$
\end{enumerate}
\end{definition}

\begin{definition}[Hilbert Space]
A \emph{Hilbert space} is a complete inner product space (complete with respect to the norm $\|x\| = \sqrt{\langle x, x \rangle}$).
\end{definition}

\begin{theorem}[Riesz Representation]
For any continuous linear functional $\phi$ on a Hilbert space $H$, there exists a unique $y \in H$ such that $\phi(x) = \langle x, y \rangle$ for all $x \in H$.
\end{theorem}

\subsection{Operators}

\begin{definition}[Self-adjoint Operator]
A linear operator $T: D(T) \to H$ on a Hilbert space is \emph{self-adjoint} if $D(T) = D(T^*)$ and $T = T^*$, where $T^*$ is the adjoint satisfying $\langle Tx, y \rangle = \langle x, T^*y \rangle$.
\end{definition}

\begin{theorem}[Spectral Theorem]
Every self-adjoint operator $T$ on a Hilbert space has a spectral decomposition:
\[
T = \int_{\mathbb{R}} \lambda \, dE_{\lambda}
\]
where $\{E_{\lambda}\}$ is the spectral family of projection operators.
\end{theorem}

\section{Representation Theory}

\subsection{Group Representations}

\begin{definition}[Group Representation]
A \emph{representation} of a group $G$ on a vector space $V$ is a homomorphism $\rho: G \to GL(V)$.
\end{definition}

\begin{definition}[Irreducible Representation]
A representation is \emph{irreducible} if the only $G$-invariant subspaces are $\{0\}$ and $V$.
\end{definition}

\begin{theorem}[Schur's Lemma]
If $\rho_1: G \to GL(V_1)$ and $\rho_2: G \to GL(V_2)$ are irreducible representations and $T: V_1 \to V_2$ is a $G$-equivariant linear map, then either $T = 0$ or $T$ is an isomorphism.
\end{theorem}

\subsection{Characters}

\begin{definition}[Character]
The \emph{character} of a representation $\rho$ is the function $\chi_{\rho}(g) = \text{tr}(\rho(g))$.
\end{definition}

\begin{theorem}[Orthogonality of Characters]
For finite groups, if $\chi_1$ and $\chi_2$ are characters of irreducible representations:
\[
\langle \chi_1, \chi_2 \rangle = \frac{1}{|G|} \sum_{g \in G} \chi_1(g) \overline{\chi_2(g)} = \begin{cases}
1 & \text{if representations are equivalent} \\
0 & \text{otherwise}
\end{cases}
\]
\end{theorem}

\section{Differential Geometry}

\subsection{Riemannian Manifolds}

\begin{definition}[Riemannian Metric]
A \emph{Riemannian metric} on a manifold $M$ is a smooth assignment of an inner product $g_p$ to each tangent space $T_pM$.
\end{definition}

\begin{definition}[Laplace-Beltrami Operator]
On a Riemannian manifold, the \emph{Laplace-Beltrami operator} is:
\[
\Delta f = \frac{1}{\sqrt{|g|}} \partial_i \left( \sqrt{|g|} g^{ij} \partial_j f \right)
\]
where $g^{ij}$ is the inverse metric tensor and $|g|$ is the determinant.
\end{definition}

\subsection{Hyperbolic Geometry}

\begin{definition}[Upper Half-Plane Model]
The \emph{hyperbolic upper half-plane} is $\mathbb{H} = \{z \in \mathbb{C} : \Im(z) > 0\}$ with metric:
\[
ds^2 = \frac{dx^2 + dy^2}{y^2}
\]
\end{definition}

\begin{theorem}[Isometry Group]
The isometry group of $\mathbb{H}$ is $PSL_2(\mathbb{R})$ acting by Möbius transformations:
\[
z \mapsto \frac{az + b}{cz + d}, \quad \begin{pmatrix} a & b \\ c & d \end{pmatrix} \in SL_2(\mathbb{R})
\]
\end{theorem}

\section{Modular Forms}

\begin{definition}[Modular Form]
A \emph{modular form} of weight $k$ for $SL_2(\mathbb{Z})$ is a holomorphic function $f: \mathbb{H} \to \mathbb{C}$ satisfying:
\begin{enumerate}
\item $f\left(\frac{az + b}{cz + d}\right) = (cz + d)^k f(z)$ for all $\begin{pmatrix} a & b \\ c & d \end{pmatrix} \in SL_2(\mathbb{Z})$
\item $f$ is holomorphic at infinity (has a Fourier expansion $f(z) = \sum_{n=0}^{\infty} a_n e^{2\pi i nz}$)
\end{enumerate}
\end{definition}

\begin{theorem}[Dimension Formula]
The dimension of the space $M_k$ of modular forms of weight $k$ is:
\[
\dim M_k = \begin{cases}
0 & \text{if } k < 0 \text{ or } k \text{ odd} \\
\lfloor k/12 \rfloor & \text{if } k \equiv 2 \pmod{12} \\
\lfloor k/12 \rfloor + 1 & \text{otherwise}
\end{cases}
\]
\end{theorem}

\section{Essential Inequalities}

\begin{theorem}[Cauchy-Schwarz Inequality]
For any inner product space:
\[
|\langle x, y \rangle| \leq \|x\| \cdot \|y\|
\]
\end{theorem}

\begin{theorem}[Jensen's Inequality]
For a convex function $\phi$ and probability measure $\mu$:
\[
\phi\left(\int f \, d\mu\right) \leq \int \phi(f) \, d\mu
\]
\end{theorem}

\begin{theorem}[Phragmén-Lindelöf Principle]
If $f$ is holomorphic in a sector and continuous on its closure, with $|f(z)| \leq M$ on the boundary, and if $f$ doesn't grow too rapidly, then $|f(z)| \leq M$ throughout the sector.
\end{theorem}

This appendix provides the essential background for understanding the main text. Readers requiring more detail should consult standard texts in complex analysis, number theory, and functional analysis.
