% Appendix C: Historical Timeline of the Riemann Hypothesis

This timeline traces the major developments, attempts, and discoveries related to the Riemann Hypothesis from its inception to the present day.

\section*{19th Century: Foundations}

\begin{itemize}
\item \textbf{1859}: Bernhard Riemann publishes \emph{\"Uber die Anzahl der Primzahlen unter einer gegebenen Gr\"o{\ss}e}, introducing the hypothesis that all non-trivial zeros of $\zeta(s)$ have real part $1/2$.

\item \textbf{1885}: Thomas Stieltjes claims to have proved the Mertens conjecture (which would imply RH) in correspondence with Hermite. The proof never appears.

\item \textbf{1896}: Jacques Hadamard and Charles Jean de la Vall\'ee Poussin independently prove the Prime Number Theorem using properties of $\zeta(s)$.

\item \textbf{1897}: Franz Mertens publishes his conjecture that $|M(x)| < \sqrt{x}$, unaware of Stieltjes's earlier claim.
\end{itemize}

\section*{Early 20th Century: First Breakthroughs}

\begin{itemize}
\item \textbf{1900}: David Hilbert includes RH as Problem 8 in his famous list of 23 problems for the 20th century.

\item \textbf{1908}: Ernst Lindel\"of proposes the Lindel\"of hypothesis: $\zeta(1/2 + it) = O(t^\epsilon)$ for any $\epsilon > 0$.

\item \textbf{1912}: von Sterneck verifies the Mertens conjecture computationally up to $x < 10^9$.

\item \textbf{1914}: 
  \begin{itemize}
  \item G.H. Hardy proves infinitely many zeros lie on the critical line
  \item J.E. Littlewood proves $\pi(x) - \Li(x)$ changes sign infinitely often
  \end{itemize}

\item \textbf{1921}: Hardy and Littlewood prove that $\gg T$ zeros up to height $T$ lie on the critical line.

\item \textbf{1922-1923}: Hardy and Littlewood formulate their conjectures on prime k-tuples and prime gaps.

\item \textbf{1932}: Carl Ludwig Siegel discovers the Riemann-Siegel formula in Riemann's unpublished manuscripts.

\item \textbf{1933}: Stanley Skewes proves that if RH is true, then the first sign change of $\pi(x) - \Li(x)$ occurs before $e^{e^{e^{79}}}$ (first Skewes number).

\item \textbf{1942}: Atle Selberg improves Hardy-Littlewood to show $\gg T \log T$ zeros on the critical line (a positive proportion).

\item \textbf{1948}: Andr\'e Weil proves the Riemann hypothesis for curves over finite fields.
\end{itemize}

\section*{Mid-20th Century: New Approaches}

\begin{itemize}
\item \textbf{1950}: Atle Selberg develops the Selberg trace formula, connecting spectral theory to number theory.

\item \textbf{1955}: Skewes proves unconditionally that the first sign change occurs before $10^{10^{10^{963}}}$ (second Skewes number).

\item \textbf{1960s}: Louis de Branges develops his theory of Hilbert spaces of entire functions as an approach to RH.

\item \textbf{1966}: H.J.J. te Riele begins systematic computational verification of RH zeros.

\item \textbf{1972}: Hugh Montgomery discovers the pair correlation conjecture connecting zero spacing to random matrix theory.

\item \textbf{1973}: Freeman Dyson recognizes Montgomery's result as matching GUE random matrix statistics.

\item \textbf{1974}: Pierre Deligne proves the Weil conjectures, establishing RH for varieties over finite fields.

\item \textbf{1974}: Norman Levinson proves that at least 1/3 of the zeros are on the critical line.

\item \textbf{1985}: Andrew Odlyzko and Herman te Riele disprove the Mertens conjecture using the LLL lattice reduction algorithm.

\item \textbf{1986}: Z.C. Feng improves critical line result to more than 1/3 of zeros.

\item \textbf{1989}: Brian Conrey proves that at least 40\% of zeros lie on the critical line.
\end{itemize}

\section*{Late 20th Century: Computational Era}

\begin{itemize}
\item \textbf{1991}: Odlyzko computes millions of zeros to high precision, confirming Montgomery's pair correlation.

\item \textbf{1998}: The ZetaGrid distributed computing project begins systematic verification of RH.

\item \textbf{2000}: Clay Mathematics Institute lists RH as one of seven Millennium Prize Problems with \$1 million reward.

\item \textbf{2001}: First $10^{10}$ zeros verified to lie on the critical line.
\end{itemize}

\section*{21st Century: Modern Developments}

\begin{itemize}
\item \textbf{2004}: Xavier Gourdon verifies the first $10^{13}$ zeros using the Odlyzko-Sch\"onhage algorithm.

\item \textbf{2005}: Conrey and Li identify gaps in de Branges's approach to RH.

\item \textbf{2011}: Marek Wolf extends Skewes number concept to twin primes.

\item \textbf{2012}: Z.C. Feng improves the proportion of zeros on critical line to 41.28\%.

\item \textbf{2018}: Michael Atiyah claims a proof of RH using the Todd function; the claim is not accepted.

\item \textbf{2020}: 
  \begin{itemize}
  \item Brad Rodgers and Terence Tao prove the de Bruijn-Newman constant $\Lambda \geq 0$
  \item This shows RH is ``barely true'' if true at all
  \end{itemize}

\item \textbf{2024}: 
  \begin{itemize}
  \item Larry Guth (MIT) and James Maynard (Oxford) achieve breakthrough on zero density estimates
  \item Over $3 \times 10^{12}$ zeros computationally verified
  \end{itemize}
\end{itemize}

\section*{Key Lessons from History}

\subsection*{Failed Approaches That Advanced Understanding}

\begin{enumerate}
\item \textbf{Stieltjes (1885)}: Attempted proof via Mertens conjecture -- later shown impossible
\item \textbf{P\'olya (1920s)}: Suggested RH equivalent to eigenvalue problem -- led to Hilbert-P\'olya program
\item \textbf{de Branges (1960s-present)}: Operator theory approach -- revealed fundamental obstructions
\item \textbf{Bombieri-Garrett (2000s)}: Showed limitations of automorphic approaches
\end{enumerate}

\subsection*{Computational Milestones}

\begin{center}
\begin{tabular}{|l|l|l|}
\hline
\textbf{Year} & \textbf{Zeros Verified} & \textbf{Researcher/Project} \\
\hline
1859 & First few & Riemann (unpublished) \\
1935 & 1,041 & Titchmarsh \\
1966 & 3,500,000 & te Riele \\
1986 & $1.5 \times 10^9$ & van de Lune et al. \\
2001 & $10^{10}$ & Wedeniwski (ZetaGrid) \\
2004 & $10^{13}$ & Gourdon \\
2024 & $> 3 \times 10^{12}$ & Various projects \\
\hline
\end{tabular}
\end{center}

\subsection*{Progress on Critical Line Zeros}

\begin{center}
\begin{tabular}{|l|l|l|}
\hline
\textbf{Year} & \textbf{Result} & \textbf{Mathematician} \\
\hline
1914 & Infinitely many & Hardy \\
1921 & $\gg T$ up to height $T$ & Hardy-Littlewood \\
1942 & $\gg T \log T$ (positive proportion) & Selberg \\
1974 & $> 1/3$ of zeros & Levinson \\
1989 & $> 40\%$ of zeros & Conrey \\
2012 & $> 41.28\%$ of zeros & Feng \\
\hline
\end{tabular}
\end{center}

\section*{The Riemann Hypothesis in Context}

The timeline reveals several key patterns:

\begin{itemize}
\item \textbf{False starts are instructive}: Failed attempts (Stieltjes, Mertens) revealed the problem's depth
\item \textbf{Computational evidence can mislead}: Mertens conjecture had extensive support but was false
\item \textbf{Progress is incremental}: From Hardy's infinitely many to Feng's 41.28\% took nearly a century
\item \textbf{Interdisciplinary connections emerge slowly}: Random matrix connection discovered 1972, over a century after RH
\item \textbf{New mathematics may be required}: Each approach has revealed fundamental obstructions
\end{itemize}

The Riemann Hypothesis remains unsolved not for lack of effort or brilliance, but because it appears to require mathematical frameworks we have not yet discovered. The history suggests that the eventual proof may come from an unexpected direction, synthesizing insights from multiple failed approaches into a new mathematical paradigm.