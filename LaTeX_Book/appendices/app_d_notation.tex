% Appendix D: Notation and Conventions

\section{Number Sets and Basic Notation}

\begin{tabular}{ll}
$\mathbb{N}$ & Natural numbers $\{1, 2, 3, \ldots\}$ \\
$\mathbb{Z}$ & Integers $\{\ldots, -2, -1, 0, 1, 2, \ldots\}$ \\
$\mathbb{Q}$ & Rational numbers \\
$\mathbb{R}$ & Real numbers \\
$\mathbb{C}$ & Complex numbers \\
$\mathbb{H}$ & Upper half-plane $\{z \in \mathbb{C} : \Im(z) > 0\}$ \\
$\mathbb{F}_q$ & Finite field with $q$ elements \\
$\mathbb{Z}/n\mathbb{Z}$ & Integers modulo $n$ \\
\end{tabular}

\section{Complex Analysis Notation}

\begin{tabular}{ll}
$\Re(z)$ & Real part of $z$ \\
$\Im(z)$ & Imaginary part of $z$ \\
$|z|$ & Modulus (absolute value) of $z$ \\
$\bar{z}$ & Complex conjugate of $z$ \\
$\arg(z)$ & Argument of $z$ \\
$s = \sigma + it$ & Standard notation for complex variable \\
$\rho = \beta + i\gamma$ & Non-trivial zero of $\zeta(s)$ \\
$\res_{z=a} f(z)$ & Residue of $f$ at $z = a$ \\
$\ord_{z=a} f(z)$ & Order of zero/pole of $f$ at $z = a$ \\
\end{tabular}

\section{The Riemann Zeta Function and Related Functions}

\begin{tabular}{ll}
$\zeta(s)$ & Riemann zeta function \\
$\eta(s)$ & Dirichlet eta function \\
$\xi(s)$ & Riemann xi function $= \frac{1}{2}s(s-1)\pi^{-s/2}\Gamma(s/2)\zeta(s)$ \\
$Z(t)$ & Hardy's Z-function \\
$S(t)$ & Argument function for $\zeta(1/2 + it)$ \\
$N(T)$ & Number of zeros with $0 < \Im(\rho) \leq T$ \\
$N_0(T)$ & Number of zeros on critical line with $0 < \gamma \leq T$ \\
\end{tabular}

\section{L-Functions}

\begin{tabular}{ll}
$L(s, \chi)$ & Dirichlet L-function with character $\chi$ \\
$L(s, f)$ & L-function attached to modular form $f$ \\
$L(s, E)$ & L-function of elliptic curve $E$ \\
$\mathcal{S}$ & Selberg class \\
$\mathcal{S}^{\sharp}$ & Extended Selberg class \\
$d_F$ & Degree of $F \in \mathcal{S}$ \\
$Q_F$ & Conductor of $F \in \mathcal{S}$ \\
\end{tabular}

\section{Number-Theoretic Functions}

\begin{tabular}{ll}
$\pi(x)$ & Prime counting function \\
$\psi(x)$ & Chebyshev psi function $= \sum_{p^k \leq x} \log p$ \\
$\theta(x)$ & Chebyshev theta function $= \sum_{p \leq x} \log p$ \\
$\Li(x)$ & Logarithmic integral $= \int_2^x \frac{dt}{\log t}$ \\
$\phi(n)$ & Euler's totient function \\
$\mu(n)$ & Möbius function \\
$\Lambda(n)$ & von Mangoldt function \\
$\tau(n)$ or $d(n)$ & Number of divisors of $n$ \\
$\sigma(n)$ & Sum of divisors of $n$ \\
$\omega(n)$ & Number of distinct prime divisors of $n$ \\
$\Omega(n)$ & Number of prime divisors counted with multiplicity \\
\end{tabular}

\section{Asymptotic Notation}

\begin{tabular}{ll}
$f(x) = O(g(x))$ & $|f(x)| \leq C|g(x)|$ for some $C > 0$ and large $x$ \\
$f(x) = o(g(x))$ & $\lim_{x \to \infty} f(x)/g(x) = 0$ \\
$f(x) \sim g(x)$ & $\lim_{x \to \infty} f(x)/g(x) = 1$ \\
$f(x) = \Omega(g(x))$ & $\limsup_{x \to \infty} |f(x)|/|g(x)| > 0$ \\
$f(x) = \Theta(g(x))$ & $f(x) = O(g(x))$ and $g(x) = O(f(x))$ \\
$f(x) \ll g(x)$ & Same as $f(x) = O(g(x))$ \\
$f(x) \asymp g(x)$ & Same as $f(x) = \Theta(g(x))$ \\
\end{tabular}

\section{Summation and Product Notation}

\begin{tabular}{ll}
$\sum_{n \leq x}$ & Sum over positive integers $n \leq x$ \\
$\sum_{p}$ & Sum over prime numbers \\
$\sum_{p \leq x}$ & Sum over primes $p \leq x$ \\
$\prod_{p}$ & Product over all primes \\
$\sum'$ & Sum excluding certain terms (specified in context) \\
$\sum^*$ & Sum over primitive or reduced terms \\
$[x]$ & Greatest integer function (floor of $x$) \\
$\{x\}$ & Fractional part of $x$ \\
$(a,b)$ or $\gcd(a,b)$ & Greatest common divisor \\
$[a,b]$ or $\text{lcm}(a,b)$ & Least common multiple \\
\end{tabular}

\section{Special Functions}

\begin{tabular}{ll}
$\Gamma(s)$ & Gamma function \\
$B(x,y)$ & Beta function \\
$\psi(s)$ & Digamma function $= \Gamma'(s)/\Gamma(s)$ \\
$K_{\nu}(z)$ & Modified Bessel function of the second kind \\
$J_{\nu}(z)$ & Bessel function of the first kind \\
$\text{Ai}(x)$ & Airy function \\
$e(x)$ & $e^{2\pi i x}$ (exponential with $2\pi i$ factor) \\
$\log$ & Natural logarithm (base $e$) \\
\end{tabular}

\section{Modular Forms and Automorphic Functions}

\begin{tabular}{ll}
$SL_2(\mathbb{Z})$ & Special linear group of $2 \times 2$ integer matrices \\
$\Gamma_0(N)$ & Congruence subgroup of level $N$ \\
$M_k(\Gamma)$ & Space of modular forms of weight $k$ for $\Gamma$ \\
$S_k(\Gamma)$ & Space of cusp forms of weight $k$ for $\Gamma$ \\
$E_k$ & Eisenstein series of weight $k$ \\
$\Delta$ & Discriminant modular form \\
$j(\tau)$ & j-invariant \\
$q$ & $e^{2\pi i \tau}$ (nome) \\
\end{tabular}

\section{Operators and Transforms}

\begin{tabular}{ll}
$T_n$ & Hecke operator \\
$\mathcal{F}$ & Fourier transform \\
$\mathcal{M}$ & Mellin transform \\
$\mathcal{R}$ & Radon transform \\
$\mathcal{L}$ & Laplace transform \\
$\Delta$ & Laplacian operator \\
$\nabla$ & Gradient operator \\
$\partial/\partial z$ & Wirtinger derivative \\
$D^{\alpha}$ & Fractional derivative of order $\alpha$ \\
\end{tabular}

\section{Matrix Groups and Lie Theory}

\begin{tabular}{ll}
$GL_n$ & General linear group \\
$SL_n$ & Special linear group \\
$O(n)$ & Orthogonal group \\
$U(n)$ & Unitary group \\
$Sp_{2n}$ & Symplectic group \\
$\mathfrak{g}$ & Lie algebra (gothic font) \\
$G(\mathbb{A})$ & Adelic points of algebraic group $G$ \\
$G_{\mathbb{Q}}$ & Rational points of $G$ \\
\end{tabular}

\section{Constants}

\begin{tabular}{ll}
$\gamma$ & Euler-Mascheroni constant $\approx 0.5772\ldots$ \\
$\pi$ & Circle constant $\approx 3.14159\ldots$ \\
$e$ & Base of natural logarithm $\approx 2.71828\ldots$ \\
$\Lambda$ & de Bruijn-Newman constant ($\geq 0$) \\
$C$ & Generic positive constant (value may change) \\
$\epsilon$ & Arbitrarily small positive constant \\
$\delta$ & Small positive constant (fixed in context) \\
\end{tabular}

\section{Conventions}

\subsection{Branch Cuts}
\begin{itemize}
\item $\log z$ has branch cut along negative real axis
\item $z^s = e^{s \log z}$ inherits branch cut from $\log$
\item $\arg(z) \in (-\pi, \pi]$ unless otherwise specified
\end{itemize}

\subsection{Fourier Transform}
We use the normalization:
\[
\hat{f}(\xi) = \int_{-\infty}^{\infty} f(x) e^{-2\pi i x\xi} dx
\]

\subsection{Contour Integration}
\begin{itemize}
\item Contours are oriented counterclockwise unless specified
\item $\int_{\sigma - i\infty}^{\sigma + i\infty}$ means vertical line with $\Re(s) = \sigma$
\item Principal value integrals denoted by P.V. or dash through integral
\end{itemize}

\subsection{Empty Products and Sums}
\begin{itemize}
\item Empty products equal 1
\item Empty sums equal 0
\end{itemize}

\subsection{Prime Notation}
\begin{itemize}
\item $p$ always denotes a prime number
\item $p_n$ denotes the $n$-th prime
\item Sum/product over $p$ means over all primes
\end{itemize}

\section{Abbreviations}

\begin{tabular}{ll}
RH & Riemann Hypothesis \\
GRH & Generalized Riemann Hypothesis \\
ERH & Extended Riemann Hypothesis \\
LH & Lindelöf Hypothesis \\
GLH & Generalized Lindelöf Hypothesis \\
ABC & ABC Conjecture \\
BSD & Birch and Swinnerton-Dyer Conjecture \\
PNT & Prime Number Theorem \\
GUE & Gaussian Unitary Ensemble \\
RMT & Random Matrix Theory \\
NCG & Non-commutative Geometry \\
\end{tabular}
